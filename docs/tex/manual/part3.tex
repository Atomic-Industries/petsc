
\chapter{Advanced Options}

This section discusses options and routines that apply to all TAO
solvers and problem classes.  In particular, we focus on
convergence tests and line searches.



\section{Convergence Tests}
\label{sec:Taoconvergence}

There are many different ways to define convergence of a solver.
The methods TAO uses by default are mentioned in 
Section \ref{sec:customize}.
These methods include absolute and relative convergence tolerances as well
as a maximum number of iterations of function evaluations.
If these choices are not sufficient, 
the user can even specify a customized test\sindex{convergence tests}. 

Users can set their own customized convergence tests of the form
\begin{verbatim}
   PetscErrorCode  conv(TaoSolver tao, void *cctx);
\end{verbatim}
\noindent
The second argument
is a pointer to a structure defined by the user.
Within this routine, the solver
can be queried for the solution vector, gradient vector,
or other statistic at the current iteration through routines such as
{\tt TaoSolverGetSolutionStatus()} and  {\tt TaoSolverGetTolerances()}.

To use this convergence test within a TAO solver, use
the command \findex{TaoSolverSetConvergenceTest()}
\begin{verbatim}
      PetscErrorCode TaoSolverSetConvergenceTest(TaoSolver tao,
                                PetscErrorCode (*conv)(TaoSolver,void*),
                                void *cctx);
\end{verbatim}
\noindent
The second argument of this command is the convergence routine, and the
final argument of the convergence test routine, {\tt cctx},
denotes an optional user-defined context for private data.  
The convergence routine receives the TAO solver and this private data
structure. 
The termination flag can be set using the routine
\begin{verbatim}
   PetscErrorCode TaoSolverSetTerminationReason(TaoSolver, TaoSolverTerminationReason);
\end{verbatim}
\noindent



\section{Line Searches} \sindex{line search}
\label{sec:TaoLineSearch}

Many solvers in TAO require a line search.  While these solvers always
offer a default line search, alternative line searches can also be used
using the command line option 


Line searches must have the form:
\begin{verbatim}
    int L(TAO_SOLVER tao,TaoVec *xx,TaoVec *gg,TaoVec *dx, TaoVec *ww,
          double *f, double *step,double *gdx,int *flg,void *lsctx);
\end{verbatim}
In this routine the first argument is the TAO solver, the second argument
is the current solution vector, the third argument is the gradient at
the current point, the fourth argument is the step direction, the fourth
vector is a work vector, the fifth argument is the function value, the
sixth argument is the step length, the seventh argument is the inner
product of the gradient and direction vector used for the Armijo condition,
the eighth argument is a flag indicating success or
failure of the line search, and the last argument is a pointer to
a structure that can be used to define the line search.  When the 
routine is finished the solution vector {\tt xx}, gradient vector {\tt gg}, 
function value {\tt f}, step size {\tt step}, and {\tt flg} should be
updated to reflect the new solution.

This routine can be set with the function \findex{TaoSolverSetLineSearch()}
\begin{verbatim}
   int TaoSetLineSearch(TAO_SOLVER solver, 
                  int (*setup)(TAO_SOLVER, void*),
		  int (*options)(TAO_SOLVER,void*),
                  int (*line)(TAO_SOLVER,TaoVec*,TaoVec*,TaoVec*,TaoVec*,
                           double*,double*,double*,int*,void*),
                  int (*viewit)(TAO_SOLVER,void*),
		  int (*destroy)(TAO_SOLVER,void*),
                  void *ctx);
\end{verbatim}
\noindent
In this routine, the fourth argument is the function pointer to the line
search routine, and the seventh argument is the pointer that will be passed
to the line search routine.  The other arguments are optional function 
pointers than can be used to set up, view, and deallocate the solver.

\begin{comment}
%Three of the algorithms mentioned earlier contain a line search.
%By default, this technique employs cubic interpolation and
%backtracking \cite{dennis:83}.
%The default line search algorithm is taken from Mor\'{e} and Thuente
%\cite{more:92}. 

%Other line search methods provided by PETSc are 
%{\tt TaoQuadraticLineSearch()},
%{\tt TaoNoLineSearch()}, and {\tt TaoNoLineSearchNoNorms()},
%\findex{TaoNoLineSearch()} \findex{TaoNoLineSearchNoNorms()}
%\findex{TaoQuadraticLineSearch()}
%which can be set with the option
%{\tt -tao\_eq\_ls [cubic, quadratic, basic, basicnonorms]}. \findex{-tao_eq_ls}
%The line search routines involve several parameters, which are set
%to defaults that are reasonable for many applications.  The user
%can override the defaults by using the options
%{\tt -tao\_eq\_ls\_alpha <alpha>}, \findex{-tao_eq_ls_alpha}
%{\tt -tao\_eq\_ls\_maxstep <max>}, and \findex{-tao_eq_ls_maxstep}
%{\tt -tao\_eq\_ls\_steptol <tol>}. \findex{-tao_eq_ls_steptol}

% Again, the user can set a variety of parameters to
%control the line search; one should run a Tao program with the option 
%{\tt -help}
%for details.  Users may write their own customized line search codes
%by modeling them after one of the defaults provided.
%}
\end{comment}

\Comment{
\section{Linear Solvers}
\label{sec:TaoLinearSolvers}

One of the most computationally intensive phases of many optimization
algorithms involves the solution of systems of linear equations.  
The performance
of the linear solver may be critical to an efficient computation
of the solution.
Since linear equation solvers often have a wide variety of options 
associated with them, TAO allows the user to access the linear
solver with the command 
{\tt TaoGetLinearSolver()}\findex{TaoGetLinearSolver()}.
With access to the linear solver context, users can modify the solver
according to their preferences.  This linear solver object has a method
that allows the user to adjust convergence tolerances.  Alternatively, the
specific data structures of the linear solver can be accessed and modified
directly.  
}

%%% Local Variables: 
%%% mode: latex
%%% TeX-master: "manual_tex"
%%% End: 
