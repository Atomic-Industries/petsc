
\chapter{Advanced Options}

This section discusses options and routines that apply to all TAO
solvers and problem classes.  In particular, we focus on
convergence tests and line searches.



\section{Convergence Tests}
\label{sec:Taoconvergence}

There are many different ways to define convergence of a solver.
The methods TAO uses by default are mentioned in 
Section \ref{sec:customize}.
These methods include absolute and relative convergence tolerances as well
as a maximum number of iterations of function evaluations.
If these choices are not sufficient, 
the user can even specify a customized test\sindex{convergence tests}. 

Users can set their own customized convergence tests of the form
\begin{verbatim}
   PetscErrorCode  conv(TaoSolver tao, void *cctx);
\end{verbatim}
\noindent
The second argument
is a pointer to a structure defined by the user.
Within this routine, the solver
can be queried for the solution vector, gradient vector,
or other statistic at the current iteration through routines such as
{\tt TaoSolverGetSolutionStatus()} and  {\tt TaoSolverGetTolerances()}.

To use this convergence test within a TAO solver, use
the command \findex{TaoSolverSetConvergenceTest()}
\begin{verbatim}
      PetscErrorCode TaoSolverSetConvergenceTest(TaoSolver tao,
                                PetscErrorCode (*conv)(TaoSolver,void*),
                                void *cctx);
\end{verbatim}
\noindent
The second argument of this command is the convergence routine, and the
final argument of the convergence test routine, {\tt cctx},
denotes an optional user-defined context for private data.  
The convergence routine receives the TAO solver and this private data
structure. 
The termination flag can be set using the routine
\begin{verbatim}
   PetscErrorCode TaoSolverSetTerminationReason(TaoSolver, TaoSolverTerminationReason);
\end{verbatim}
\noindent



\section{Line Searches} \sindex{line search}
\label{sec:TaoLineSearch}

Many solvers in TAO require a line search.  While these solvers always
offer a default line search, alternative line searches can also be used
using the function
\begin{verbatim}
  PetscErrorCode TaoSolverSetDefaultLineSearchType(TaoSolver, LineSearchType);
\end{verbatim}
command line option  {\tt -tao\_ls\_type}.  Available line searches 
include Mor\'{e}-Thuente\cite{more:92}, Armijo, gpcg, and unit.

The line search routines involve several parameters, which are set
to defaults that are reasonable for many applications.  The user
can override the defaults by using the options
{\tt -tao\_ls\_maxfev <max>},
{\tt -tao\_ls\_stepmin <min>},
{\tt -tao\_ls\_stepmax <max>},
{\tt -tao\_ls\_ftol <ftol>},
{\tt -tao\_ls\_gtol <gtol>}, and
{\tt -tao\_ls\_rtol <rtol>}.

One should run a TAO program with the option 
{\tt -help}
for details.  Users may write their own customized line search codes
by modeling them after one of the defaults provided.

