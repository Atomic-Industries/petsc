% Acknowledgements for PETSc 2.0 Users Manual
%
% These are also listed on the PETSc homepage, so if you add something here
% add it to the home page also
%
\noindent {\bf Acknowledgments:}

\medskip \medskip \noindent
We especially thank Victor Eijkhout, David Keyes, and Matthew Knepley for their valuable 
comments on the 
source code, functionality, and documentation  for PETSc 2.0.  
In addition, we thank all PETSc users for
their many suggestions, bug reports, and encouragement.


\vspace{.3in}
\noindent
Some of the source code and utilities in PETSc (or software used by PETSc)
have been written by 
\begin{itemize}
  \item Mark Adams, 1996-98 (scalability features of MPIBAIJ matrices),
  \item Cameron Cooper, Fall 1995 (portions of the VecScatter routines), 
  \item Victor Eijkhout, Summer 1998 (KSP type BICG and the VecPipeline() routines), 
  \item Matt Hille, Summer 1995 (PetscView and PetscOpts), 
  \item Matthew Knepley, Summer 1997 (too much to mention),
  \item Peter Mell, Summer 1995 (portions of the DA routines),
  \item Wing-Lok Wan, Summer 1995 (the ILU portion of BlockSolve95), and
  \item Liyang Xu, Summer 1997 (the interface to PVODE)
\end{itemize}
while visiting Argonne National Laboratory or working with us.

\vspace{.3in}
\noindent
PETSc uses routines from 
\begin{itemize}
  \item BLAS
  \item LAPACK
  \item LINPACK      (matrix factorization and solve; converted to C using {\tt f2c} and then 
                      hand-optimized for small matrix sizes, for block matrix data structures),
  \item MINPACK      (sequential matrix coloring routines for finite difference Jacobian
                       evaluations; converted to C using {\tt f2c}),
  \item SPARSPAK     (matrix reordering routines, converted to C using {\tt f2c}),
  \item SPARSEKIT2 (Yousef Saad) (iludtp(), converted to C using {\tt f2c}). These routines 
                     are copyrighted by Saad under the GNU copyright, see src/mat/impls/aij/seq/gnu.
\end{itemize}


\vspace{.3in}
\noindent
PETSc interfaces to the following external software:
\begin{itemize}
  \item BlockSolve95 (for parallel ICC(0) and ILU(0) preconditioning)
  \item ESSL         (IBM's math library for fast sparse direct LU factorization)
  \item Matlab       (through a socket interface for graphics and numerical post processing 
                      of data)
  \item ParMeTiS      (parallel graph partitioner)
  \item PVODE        (parallel ODE integrator)
  \item SPAI         (for parallel sparse approximate inverse preconditiong)
\end{itemize}
These are all optional packages and do not need to be installed to use PETSc.


