%
% NOTES:  
%  - Be sure to place captions BEFORE labels in figures and tables!
%    Otherwise, numbering will be incorrect.  For example, use the following:
%       \caption{TAO Info}
%       \label{fig:taoinfo}
%  - Use \break to indicate a line break (needed to prevent long strings in
%    \tt mode from running of the page)
%
% ---------------------------------------------------------------

\chapter{Basic Usage of TAO Solvers}
%\sindex{continuous optimization}
\label{chapter:tao_solver}

TAO contains unconstrained minimization, bound constrained minimization, 
%nonlinear least squares, 
and nonlinear complementarity solvers.
The structure of these problems can differ significantly, 
but TAO has a similar interface to all of its solvers.  
Routines that most solvers have in common will be discussed in 
this chapter.
A complete list of options can be found by consulting the manual pages.
Many of the options can also be set at the command line.  These options
can also be found in manual pages or by
running a program with the {\tt -help} option.


\section{Initialize and Finalize}
The first TAO routine in any application should be {\tt TaoInitialize()}.
Most TAO programs begin with a call to
\findex{TaoInitialize()}
\begin{verbatim}
   TaoInitialize(int *argc,char ***argv,char *file_name, 
                        char *help_message);
\end{verbatim}
\noindent
This command initializes TAO, as well as MPI, PETSc, and other packages
to which TAO applications may link (if these have not yet
been initialized elsewhere).  
In particular, the arguments {\tt argc} and 
{\tt argv} are the command line arguments delivered in all C and C++
programs; these arguments initialize the options database.  
\sindex{command line arguments} The argument {\tt file\_name}
optionally indicates an alternative name for an options file, which by
default is called {\tt .petscrc} and resides in the user's home directory.

One of the last routines\findex{TaoFinalize()} that all TAO programs should 
call is 
\begin{verbatim}
   TaoFinalize();
\end{verbatim}
\noindent
This routine finalizes TAO and any other libraries that may have been
initialized during the {\tt TaoInitialize()} phase.
For example, {\tt TaoFinalize()}
calls {\tt PetscFinalize()} 
if {\tt PetscInitialize()}
began PETSc. If PETSc was initiated externally from TAO (by either
the user or another software package), then the user is
responsible for calling {\tt PetscFinalize()}. 

\section{Creation and Destruction}

A TAO solver can be created with
the command 
\findex{TaoSolverCreate()}
\begin{verbatim}
   info = TaoSolverCreate(MPI_Comm comm,TaoSolver *newsolver);
\end{verbatim}
\noindent
The first argument in this routine is an MPI communicator indicating which
processes are involved in the solution process.  In
most cases, this should be set to {\tt MPI\_COMM\_WORLD}.
The second argument in {\tt TaoSolverCreate()} is a pointer to a TAO solver
object.  This routine creates the object and returns it to the user.
The TAO object is then to be used in all TAO routines.

The routine
\findex{TaoSolverSetType()}
\begin{verbatim}
   info = TaoSolverSetType(TaoSolver tao, TaoType type);
\end{verbatim}
\noindent
can be used to set the algorithm TAO uses to solve the application.
The various types of TAO solvers and the flags that identify them 
will be discussed in the following chapters.
The solution method should be carefully chosen depending upon
the problem that is being solved.  Some solvers, for instance, are meant for
problems with no constraints, while other solvers acknowledge constraints
in the problem and solve them accordingly.
The user must also be aware of the derivative information that is available.
Some solvers require second-order information, while other solvers require
only gradient or function information.
The command line option \texttt{-tao\_method} (or equivalently 
\texttt{-tao\_type}) followed by an TAO method
will override any method specified by the second argument.
The command line option {\tt -tao\_method tao\_lmvm}, for instance,
will specify the limited memory variable metric method for unconstrained
optimization.  Note that the {\tt TaoType} variable is a string that requires
quotation marks in an application program, but quotation marks are not required
at the command line.

Each TAO solver that has been created should also be destroyed using
the command 
\findex{TaoSolverDestroy()}
\begin{verbatim}
   ierr = TaoSolverDestroy(TaoSolver tao);
\end{verbatim}
\noindent 
This routine frees the internal data structures used by the solver.


\section{Convergence}\label{sec:customize}

Although TAO and its solvers set default parameters 
that are useful
for many problems, it may be necessary for the user to modify these
parameters to change the behavior and convergence of various algorithms.

One convergence criterion for most algorithms concerns the
of digits of accuracy needed in the solution.  In particular,
one convergence test employed by TAO attempts to stop when
the error in the constraints is less than $\epsilon_{crtol}$,
 and either
\[\frac{ |f(X) - f(X^*)|}{ |f(X)| + 1} \leq \epsilon_{frtol}
\;\textnormal{or}\;
f(X) - f(X^*)  \leq \epsilon_{fatol}, \]
where $X^*$ is the current approximation to the true solution $X$.
$X$ is unknown, so TAO estimates $f(X) - f(X^*)$ with either 
the square of the norm of the gradient or the duality gap.
A relative tolerance of $\epsilon_{frtol}=0.01$ indicates that two
significant digits are desired in the objective function.
Each solver sets its own  convergence tolerances, but they can
be changed using the routine
{\tt TaoSolverSetTolerances()}\findex{TaoSolverSetTolerances()} 
\sindex{convergence tests}. 
Another set of convergence tolerances 
terminate the solver when the norm of the gradient function
(or Lagrangian function for bound-constrained problems)
is sufficiently close to zero.

Other stopping criteria include a minimum trust region radius or 
a maximum number of iterations.  These parameters can be set with
the routines {\tt TaoSolverSetTrustRegionTolerance()}\sindex{trust region}\findex{TaoSolverSetTrustRegionTolerance}
and {\tt TaoSolverSetMaximumIterations()}\findex{TaoSolverSetMaximumIterations()}.
Similarly, a maximum number of function evaluations can be set 
with the command 
{\tt TaoSolverSetMaximumFunctionEvaluations()}
\findex{TaoSolverSetMaximumFunctionEvaluations()}.

\section{Viewing Solutions}

The routine
\begin{verbatim}
   ierr =  TaoSolverSolve(TaoSolver tao)
\end{verbatim}
\noindent
will apply the solver to the application that has been created by the user.

To see parameters and performance statistics for the solver, the
routine
\begin{verbatim}
   ierr = TaoSolverView(TaoSolver tao)
\end{verbatim}
can be used.  This routine will display to standard output the number
of function evaluations need by the solver and other information
specific to the solver.  This same output can be produced by using the 
command line option \tt{-tao\_view}.

The progress of the optimization solver can be monitored with
the runtime option {\tt -tao\_monitor}.  Although monitoring routines
can be customized, the default monitoring routine will print out 
several relevant statistics to the screen.

The user also has access to information about the current solution.
The current iteration number, objective function value, gradient
norm, infeasibility norm, and step length 
can be retrieved with the command 
\findex{TaoGetSolutionStatus()}
\begin{verbatim}
   ierr = TaoSolverGetSolutionStatus(TaoSolver tao, PetscInt *iterate, 
                            PetscReal *f, PetscReal *gnorm, PetscReal *cnorm,
                            PetscReal *xdiff, TaoSolverTerminationReason *reason)
\end{verbatim}
\noindent
The last argument returns
a code that indicates the reason that the solver terminated.  Positive 
numbers indicate that a solution has been found, while negative numbers
indicate a failure.  A list of reasons can be found in the manual page
for {\tt TaoSolverGetConvergedReason()}.

The user set
vectors containing the solution before solving
the problem, but pointers to these vectors can also be retrieved with the
command {\tt TaoGetSolution()}\findex{TaoGetSolution()}.
Dual variables and other relevant information are also available. 
This information can be obtained during
user-defined routines such as a function evaluation and customized
monitoring routine, or after the solver has terminated.

%%% Local Variables: 
%%% mode: latex
%%% TeX-master: "manual_tex"
%%% End: 
