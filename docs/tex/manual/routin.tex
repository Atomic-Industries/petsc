%
% Be sure to keep either no spaced between headings or use \noindent
%

\markright{Summary of Routines}

% k is used to put sorting information in a regular place, and to 
% hold the table lines.  Note that we do not use a tabular environment
% so that a multipage table can be generated.
% The first argument is used for sorting purposes only
\def\tightboxit#1{\vbox{\hsize=6.6in \hrule\hbox{\vrule\vbox{\hbox{#1}}\vrule
\hss}\hrule}}
\def\k#1#2#3{\tightboxit{
    \parindent=0pt\hskip.3in\begin{tabular}{l} #2 \\ #3 \end{tabular}}}
\def\CoDe#1{
    \parbox[t]{6.0in}{\raggedright \parindent=-0.4in {\bf #1}}\break}
\def\DeFn#1{
    \parbox[t]{6.0in}{\raggedright #1 \vskip2pt}\break}


\noindent {\Large \bf Routine Prefixes}

\vspace{.2cm}

\begin{tabular}{llll}
 Vec&Vector&Mat&Matrix routines\\
SLES&Linear equation solvers& KSP&Krylov subspace methods\\
PC&Preconditioners&IS&Index sets\\
SNES & Nonlinear equation solvers&DA& Distributed arrays \\
TS   & timesteppers & Options & Options database \\
Petsc & Miscellaneous system & Draw & Drawing graphics routines\\
AO & Application & PLog & Profiling functions \\
\end{tabular}

\section{Vector Routines}

\noindent
Data Structures: 
\begin{itemize}
\item Vec - a vector of any type
\item VecScatter - an object used in scatter routines, which allows
      reuse of communication information between different (but identical)
      scatters
\item AO - application orderings, which provide mappings between user orderings
           and PETSc orderings
\end{itemize}
Norms:
\begin{itemize}
\item NORM\_1
\item NORM\_2
\item NORM\_INFINITY or NORM\_MAX  
\end{itemize}
InsertMode:
\begin{itemize}
\item INSERT\_VALUES
\item ADD\_VALUE
\end{itemize}
Runtime Options:
\begin{itemize}
\item -vec\_mpi
\item -vec\_view
\item -vec\_view\_draw
\item -vec\_view\_draw\_lg
\item -vec\_view\_matlab
\end{itemize}

\noindent {  \#include ``vec.h''}

{\small
\noindent
\input ../rsum/ssum1
}

\section{Matrix Routines}

\noindent
Data Structures: 
\begin{itemize}
\item Mat - a matrix of any type (including matrix-free)
\end{itemize}
Matrix options:
\begin{itemize}
\item MAT\_ROW\_ORIENTED - inserts are done with row-oriented blocks
\item MAT\_COLUMN\_ORIENTED - inserts are done with column-oriented blocks
\item MAT\_ROWS\_SORTED - row indices in the inserted block are sorted
\item MAT\_COLUMNS\_SORTED - column indices in the inserted block are sorted
\item MAT\_NO\_NEW\_NONZERO\_LOCATIONS - additional inserts will not be allowed if they
      generate a new non-zero.
\item MAT\_YES\_NEW\_NONZERO\_LOCATIONS - additional inserts will be allowed
\item MAT\_SYMMETRIC\_MATRIX - matrix is known to be symmetric
\item MAT\_STRUCTURALLY\_SYMMETRIC\_MATRIX - matrix is known to be symmetricin nonzero structure
\end{itemize}
Orderings:
\begin{itemize}
\item ORDER\_NATURAL - Natural
\item ORDER\_ND - Nested Dissection
\item ORDER\_1WD - One-Way Dissection 
\item ORDER\_RCM - Reverse Cuthill-McGee
\item ORDER\_QMD - Quotient Minimum Degree
\end{itemize}
Norms:
\begin{itemize}
\item NORM\_1
\item NORM\_2
\item NORM\_FROBENIUS
\item NORM\_INFINITY or NORM\_MAX  
\end{itemize}
Options to MatAssemblyXXX():
\begin{itemize}
\item MAT\_FLUSH\_ASSEMBLY - intermediate matrix assembly
\item MAT\_FINAL\_ASSEMBLY - final matrix assembly
\end{itemize}
Options to MatGetInfo():
\begin{itemize}
\item MAT\_LOCAL
\item MAT\_GLOBAL\_MAX
\item MAT\_GLOBAL\_SUM
\end{itemize}
MatType:
\begin{itemize}
\item MATSAME - same format as current (e.g., used for {\tt MatConvert()})
\item MATSEQAIJ - sequential sparse row (AIJ) format
\item MATMPIAIJ - parallel sparse row (AIJ) format
\item MATSEQDENSE - sequential dense format
\item MATMPIDENSE - parallel dense format
\item MATMPIROWBS - parallel row-based format compatible with BlockSolve95
\item MATSEQBAIJ - sequential sparse block row (BAIJ) format
\item MATMPIBAIJ - parallel sparse block row (BAIJ) format
\item MATSEQBDIAG - sequential block diagonal format
\item MATMPIBDIAG - parallel block diagonal format
\end{itemize}
MatGetSubMatrixCall:
\begin{itemize}
\item MAT\_INITIAL\_MATRIX
\item MAT\_REUSE\_MATRIX
\end{itemize}
Runtime Options:
\begin{itemize}
\item -mat\_aij
\item -mat\_aij\_dxml
\item -mat\_aij\_essl
\item -mat\_aij\_inode\_limit [limit]
\item -mat\_aij\_no\_inode
\item -mat\_aij\_oneindex
\item -mat\_aij\_superlu
\item -mat\_baij
\item -mat\_bdiag
\item -mat\_bdiag\_diags [diag\_number\_1,diag\_number\_2,...]
\item -mat\_bdiag\_ndiag [number\_of\_diagonals]
\item -mat\_block\_size [size]
\item -mat\_coloring
\item -mat\_dense
\item -mat\_ilu\_fill [fill]
\item -mat\_lu\_fill [fill]
\item -mat\_lu\_pivotthreshold [threshold]
\item -mat\_mpiaij
\item -mat\_mpibaij
\item -mat\_mpibdiag
\item -mat\_mpidense
\item -mat\_mpirowbs
\item -mat\_no\_unroll
\item -mat\_order [order]
\item -mat\_rowbs\_no\_inode
\item -mat\_seqaij
\item -mat\_seqbaij
\item -mat\_seqbdiag
\item -mat\_seqdense
\item -mat\_view
\item -mat\_view\_draw
\item -mat\_view\_info
\item -mat\_view\_info\_detailed
\item -mat\_view\_matlab
\item -matload\_bdiag\_diags 
\item -matload\_block\_size [size]
\item -matload\_ignore\_info
\end{itemize}

\noindent {  \#include ``mat.h''}

{\small
\noindent
\input ../rsum/ssum2
}

\section{Simplified Linear Solvers}

\noindent
Data Structures:
\begin{itemize}
\item SLES - linear equation solver context
\end{itemize}
Runtime Options:
\begin{itemize}
\item -sles\_view
\end{itemize}

\noindent {  \#include ``sles.h''}

{\small
\noindent
\input ../rsum/ssum3
}

\section{Preconditioners}

\noindent
Data Structures:
\begin{itemize}
\item PC - preconditioner context
\end{itemize}
Available Methods: 
\begin{itemize}
\item PCNONE - null preconditioner
\item PCJACOBI - Jacobi
\item PCSOR - SOR/SSOR and other variants
\item PCEISENSTAT - SOR using the Eisenstat trick
\item PCBJACOBI - block Jacobi
\item PCASM - additive overlapping Schwarz
\item PCILU - incomplete LU
\item PCICC - incomplete Cholesky
\item PCBGS - block Gauss-Seidel
\item PCMG - multigrid
\item PCSHELL - user-defined shell preconditioner
\item PCLU - LU (direct solver)
\end{itemize}
MGType:
\begin{itemize}
\item MGMULTIPLICATIVE
\item  MGADDITIVE
\item  MGFULL
\item  MGKASKADE
\end{itemize}
Runtime Options:
\begin{itemize}
\item -pc\_type [jacobi,bjacobi,sor,eisenstat,ilu,icc,asm,bgs,mg,lu,shell,none]
\item -pc\_asm\_blocks [blocks]
\item -pc\_asm\_overlap [overlap]
\item -pc\_bgs\_blocks [blocks]
\item -pc\_bgs\_symmetric
\item -pc\_bgs\_truelocal
\item -pc\_bjacobi\_blocks [blocks]
\item -pc\_bjacobi\_truelocal
\item -pc\_eisenstat\_diagonal\_scaling
\item -pc\_eisenstat\_omega [omega]
\item -pc\_icc\_factorpointwise
\item -pc\_ilu\_factorpointwise
\item -pc\_ilu\_in\_place
\item -pc\_ilu\_levels [levels]
\item -pc\_ilu\_preserve\_row\_sums
\item -pc\_ilu\_reuse\_fill
\item -pc\_ilu\_reuse\_reordering
\item -pc\_ilu\_use\_drop\_tolerance
\item -pc\_lu\_in\_place
\item -pc\_mg\_cycles [cycles]
\item -pc\_mg\_levels [levels]
\item -pc\_mg\_method [method]
\item -pc\_mg\_smoothdown
\item -pc\_mg\_smoothup
\item -pc\_sor\_backward
\item -pc\_sor\_its [iterations]
\item -pc\_sor\_local\_backward
\item -pc\_sor\_local\_forward
\item -pc\_sor\_local\_symmetric
\item -pc\_sor\_omega [omega]
\item -pc\_sor\_symmetric
\end{itemize}

\noindent {  \#include ``pc.h''}

{\small
\noindent
\input ../rsum/ssum4
}

\section{Krylov Subspace Methods}

\noindent
Data Structures:
\begin{itemize}
\item KSP - Krylov space sover context
\end{itemize}
Runtime Options:
\begin{itemize}
\item -ksp\_type [cg,cgs,bcgs,gmres,tcqmr,tfqmr,cr,richardson,chebyshev,lsqr,qcg,preonly]
\item -ksp\_atol [absolute\_tolerance]
\item -ksp\_bsmonitor
\item -ksp\_cg\_Hermitian
\item -ksp\_cg\_symmetric
\item -ksp\_compute\_eigenvalues
\item -ksp\_compute\_eigenvalues\_explicitly
\item -ksp\_divtol [divergence\_tolerance]
\item -ksp\_eigen
\item -ksp\_gmres\_irorthog
\item -ksp\_gmres\_preallocate
\item -ksp\_gmres\_restart [restart\_number]
\item -ksp\_gmres\_unmodifiedgramschmidt
\item -ksp\_left\_pc
\item -ksp\_max\_it [maximum\_iterations]
\item -ksp\_monitor
\item -ksp\_plot\_eigenvalues
\item -ksp\_plot\_eigenvalues\_explicitly
\item -ksp\_preres
\item -ksp\_richardson\_scale
\item -ksp\_right\_pc
\item -ksp\_rtol [relative\_tolerance]
\item -ksp\_singmonitor
\item -ksp\_smonitor
\item -ksp\_symmetric\_pc
\item -ksp\_truemonitor
\item -ksp\_type
\item -ksp\_xmonitor
\item -ksp\_xtruemonitor
\end{itemize}
KSPType: KSPRICHARDSON, KSPCHEBYCHEV, KSPCG, KSPGMRES, 
         KSPTCQMR, KSPBCGS, KSPCGS, KSPTFQMR, KSPCR, KSPLSQR, KSPQCG,
         KSPPREONLY

\noindent {\#include ``ksp.h''}

{\small
\noindent
\input ../rsum/ssum5
}

\section{Nonlinear Solvers}

\noindent
Data Structures:
\begin{itemize}
\item SNES - nonlinear solver context
\end{itemize}
Available Methods: 
\begin{itemize}
\item SNES\_EQ\_LS - line search for systems of nonlinear equations
\item SNES\_EQ\_TR - trust region for systems of nonlinear equations
\item SNES\_UM\_LS - line search for unconstrained minimization
\item SNES\_UM\_TR - trust region for unconstrained minimization
\end{itemize}
Runtime Options:
\begin{itemize}
\item -snes\_type [ls,tr,umtr,umls,test]
\item -snes\_atol [absolute\_tolerance]
\item -snes\_eq\_ls
\item -snes\_eq\_ls\_alpha [alpha]
\item -snes\_eq\_ls\_maxstep [maxstep]
\item -snes\_eq\_ls\_steptol [steptol]
\item -snes\_eq\_tr\_delta0 [delta0]
\item -snes\_eq\_tr\_delta1 [delta1]
\item -snes\_eq\_tr\_delta2 [delta2]
\item -snes\_eq\_tr\_delta3 [delta3]
\item -snes\_eq\_tr\_eta [eta]
\item -snes\_eq\_tr\_mu [mu]
\item -snes\_eq\_tr\_sigma [sigma]
\item -snes\_fd
\item -snes\_fmin
\item -snes\_ksp\_ew\_alpha [alpha]
\item -snes\_ksp\_ew\_alpha2 [alpha2]
\item -snes\_ksp\_ew\_conv [conv]
\item -snes\_ksp\_ew\_gamma [gamma]
\item -snes\_ksp\_ew\_rtol0 [rtol0]
\item -snes\_ksp\_ew\_rtolmax [rtolmax]
\item -snes\_ksp\_ew\_threshold [threshold]
\item -snes\_ksp\_ew\_version [version]
\item -snes\_max\_funcs [maximum\_function\_evaluations]
\item -snes\_max\_it [maximum\_iterations]
\item -snes\_mf
\item -snes\_mf\_err [err]
\item -snes\_mf\_operator
\item -snes\_mf\_umin [minimum\_value]
\item -snes\_monitor
\item -snes\_rtol [relative\_tolerance]
\item -snes\_smonitor
\item -snes\_stol [step\_tolerance]
\item -snes\_test\_display 
\item -snes\_trtol [trust\_region\_tolerance]
\item -snes\_um\_delta0
\item -snes\_um\_eta1
\item -snes\_um\_eta2
\item -snes\_um\_eta3
\item -snes\_um\_eta4
\item -snes\_um\_factor1
\item -snes\_um\_ls\_ftol
\item -snes\_um\_ls\_gamma\_factor
\item -snes\_um\_ls\_gtol
\item -snes\_um\_ls\_maxfev
\item -snes\_um\_ls\_rtol
\item -snes\_um\_ls\_stepmax
\item -snes\_um\_ls\_stepmin
\item -snes\_view
\item -snes\_xmonitor
\end{itemize}

\noindent {\#include ``snes.h''}

{\small
\noindent
\input ../rsum/ssumn
}

\section{Timestepping, ODE Solvers}

\noindent
Available Methods: 
\begin{itemize}
\item TS\_EULER - Euler method
\item TS\_BEULER - backward Euler method
\item TS\_PSEUDO\_POSITION\_INDEPENDENT\_TIMESTEP - pseudo-transient timestep variant 1
\item TS\_PSEUDO\_POSITION\_DEPENDENT\_TIMESTEP - pseudo-transient timestep variant 2
TSType: 
\end{itemize}
Data Structures:
\begin{itemize}
\item TS - timestepping context
\end{itemize}
TSProblemType:
\begin{itemize}
\item TS\_LINEAR
\item TS\_NONLINEAR
\end{itemize}
Runtime Options:
\begin{itemize}
\item -ts\_max\_steps [steps]
\item -ts\_monitor
\item -ts\_pseudo\_increment [increment]
\item -ts\_type
\item -ts\_view
\end{itemize}

\noindent {\#include ``ts.h''}

{\small
\noindent
\input ../rsum/ssumt
}

\section{Index Sets, Distributed Arrays, and Application Orderings}

\noindent
Data Structures:
\begin{itemize}
\item IS - an index set
\item DA - a distributed array
\item AO - an application ordering
\end{itemize}
DAPeriodicType:
\begin{itemize}
\item DA\_NONPERIODIC
\item  DA\_XPERIODIC
\item  DA\_YPERIODIC
\item  DA\_XYPERIODIC
\item  DA\_XYZPERIODIC
\item  DA\_XZPERIODIC
\item  DA\_YZPERIODIC
\item DA\_ZPERIODIC
\end{itemize}
DAStencilType:
\begin{itemize}
\item DA\_STENCIL\_STAR
\item DA\_STENCIL\_BOX 
\end{itemize}
Runtime Options:
\begin{itemize}
\item -ao\_view
\item -da\_partition\_blockcomm
\item -da\_partition\_nodes\_at\_end
\item -da\_view
\end{itemize}

\noindent {\#include ``is.h''} \\
\noindent {\#include ``da.h''}

{\small
\noindent
\input ../rsum/ssum6
}

\section{Utility and System Routines}

\noindent
Runtime Options:
\begin{itemize}
\item -debugger\_nodes [nodes]
\item -debugger\_pause [seconds]
\item -fp\_trap
\item -help (or \item -h)
\item -log\_history
\item -mpidump
\item -no\_signal\_handler
\item -on\_error\_abort
\item -on\_error\_attach\_debugger
\item -on\_error\_stop
\item -optionsleft
\item -optionstable
\item -trdebug
\item -trdump
\item -trinfo
\item -trmalloc
\item -trmalloc\_off
\item -version (or -v)
\end{itemize}

\noindent {  \#include ``sys.h''} \\
{  \#include ``options.h''} \\

{\small
\noindent
\input ../rsum/ssum7
}

\section{Viewers}

\noindent
Default Viewers:
\begin{itemize}
\item VIEWER\_STDOUT\_WORLD
\item VIEWER\_STDOUT\_SELF
\item VIEWER\_DRAWX\_WORLD
\item VIEWER\_DRAWX\_SELF
\item VIEWER\_MATLAB\_WORLD
\end{itemize}
Format options:
\begin{itemize}
\item ASCII\_FORMAT\_DEFAULT - default
\item ASCII\_FORMAT\_MATLAB - Matlab format
\item  ASCII\_FORMAT\_IMPL - implementation-specific format
      (which is, in many cases, the same as the default)
\item ASCII\_FORMAT\_INFO - basic information about object
\item ASCII\_FORMAT\_INFO\_DETAILED - more detailed info about object
\item ASCII\_FORMAT\_COMMON - identical output format for
       all objects of a particular type
\item BINARY\_FORMAT\_NATIVE - store the object to disk in the format it
      is in. This currently works only for dense matrices.
\end{itemize}

{\small
\noindent
\input ../rsum/ssumv
}

\section{Profiling}
Runtime Options:
\begin{itemize}
\item -log [filename]
\item -log\_summary
\item -log\_all [filename]
\item -log\_mpe [filename]
\end{itemize}

{\small
\noindent
\input ../rsum/ssump
}

\section{Graphics Routines}

\noindent
Data Structures:
\begin{itemize}
\item Draw - a drawing surface, probably a window.
\item DrawAxis - a two-dimensional line graph axis.
\item DrawLG - a two-dimensional line graph.
\end{itemize}

\noindent {  \#include ``draw.h''}

{\small
\noindent
\input ../rsum/ssumx
}

