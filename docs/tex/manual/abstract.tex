% Abstract for PETSc 2.0 Users Manual

%
%   Next line temp removed
%
\noindent {\bf Abstract:} 

\medskip \medskip
This manual describes the use of PETSc 2.0 for the numerical solution
of partial differential equations and related problems 
on high-performance computers.  The
Portable, Extensible Toolkit for Scientific Computation (PETSc) is a
suite of data structures and routines that provide the building
blocks for the implementation of large-scale application codes on parallel
(and serial) computers.  PETSc 2.0 uses the MPI standard for all
message-passing communication.

PETSc includes an expanding suite of parallel linear and nonlinear
equation solvers and unconstrained minimization modules that may be
used in application codes written in Fortran, C, and C++.  PETSc
provides many of the mechanisms needed within parallel application
codes, such as simple parallel matrix and vector assembly routines
that allow the overlap of communication and computation.  In addition,
PETSc includes growing support for distributed arrays.  The library is
organized hierarchically, enabling users to employ the level of
abstraction that is most appropriate for a particular problem. By
using techniques of object oriented programming, PETSc provides
enormous flexibility for users.

PETSc is a sophisticated set of software tools; as such, for some
users it initially has a much steeper learning curve than a
simple subroutine library. In particular, for individuals without some
computer science background or experience programming in C,
Pascal, or C++, it may require a significant amount of time to take full
advantage of the features that enable efficient software use.
However, the power of the PETSc design and
the algorithms it incorporates make the efficient implementation of
many application codes much simpler than ``rolling them'' yourself.
For many simple (or even relatively complicated) tasks a package such as
Matlab is often the best tool; PETSc is not intended for the classes
of problems for which effective Matlab code can be written.

Since PETSc is still under development, small changes in usage and
calling sequences of PETSc routines will continue to occur.  Although
keeping one's code up to date can be somewhat annoying,
all PETSc users will be rewarded in the long run with a cleaner,
better designed, and easier-to-use interface.
