% Abstract for PETSc 2.0 Users Manual

\noindent {\bf Abstract:} 

\medskip \medskip
This manual describes the use of PETSc 2.0 for the numerical solution
of partial differential equations on high-performance computers.  The
Portable, Extensible Toolkit for Scientific computation (PETSc), is a
suite of data structures and routines that provide the building
blocks for the implementation of large-scale application codes on parallel
(and serial) computers.  PETSc 2.0 uses the MPI standard for all
message-passing communication.

PETSc includes an expanding suite of parallel linear and nonlinear
equation solvers that are easily used in application codes written in
Fortran, C, and C++.  PETSc provides many of the mechanisms needed
within parallel application codes, such as simple parallel matrix and
vector assembly routines that allow the overlap of communication and
computation.  In addition, PETSc includes growing support for
distributed arrays.  The library is organized
hierarchically, enabling users to employ the level of abstraction that
is most appropriate for a particular problem. By using techniques 
of object oriented programming, PETSc provides enormous flexibility 
for users.

PETSc is a sophisticated set of software tools, as such, for some
users this may mean it initially has a steeper learning curve then a
simple subroutine library. However, the power of the PETSc design and
the algorithms it incorporates make the efficient implementation of
many application codes much simpler then ``rolling them'' yourself.

PETSc is still under development, so that small changes in usage and
calling sequences of PETSc routines will continue to occur.  Although
keeping one's code accordingly up-to-date can be somewhat annoying,
all PETSc users will be rewarded in the long run with a cleaner,
better designed, and easier-to-use interface.
