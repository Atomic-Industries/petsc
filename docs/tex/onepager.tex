% $Id: onepager.tex,v 1.19 1995/11/15 22:09:57 curfman Exp bsmith $ 
\documentstyle[epsf,sty/handpage,/home/gropp/hyper/style/anlhtext]{article}
\pagestyle{empty}
\begin{document}
\pagestyle{empty}
\title{PETSc 2.0 for MPI}
\thanks{Mathematics and Computer Science Division,
Argonne National Laboratory,
Argonne, IL 60439-4801.
This work was supported by the Mathematical,
        Information, and Computational Sciences Division subprogram of
        the Office of Computational and Technology Research,
        U.S. Department of Energy, under Contract W-31-109-Eng-38.}

\date{}
\maketitle

\newcommand{\vsp}{\vspace{-1.5mm}}

% Note:  The following classification statement was required for SC95.
%        We can zap this for other hand-outs.
% \section*{Classification} Parallel Algorithms and Software

\section*{Description}

\code{PETSc}, the Portable, Extensible Toolkit for Scientific computation,
is a suite of uni- and parallel-processor codes that are intended for
the solution of large-scale problems modeled by partial differential
equations.  \code{PETSc} employs the MPI standard for all
message-passing communication.  The code is written in a
data-structure-neutral manner to enable easy reuse and flexibility.

\code{ PETSc} is easy to use for beginners.  Moreover, its careful
design allows advanced users to have detailed control over the
solution process. \code{PETSc} includes an expanding suite of parallel
linear and nonlinear equation solvers that are easily used in
application codes written in C, C++, and Fortran.  \code{PETSc}
provides many of the mechanisms needed within parallel application
codes, such as simple parallel matrix and vector assembly routines
that allow the overlap of communication and computation.  In addition,
\code{PETSc} includes growing support for distributed arrays.

\code{PETSc} integrates a hierarchy of components, thus
enabling the user to employ the level of abstraction that is most
natural for a particular problem.  Some of the components are
\vspace{-.4cm}
\begin{itemize}
\item \code{Mat} - a suite of data structures and code
      for the manipulation of parallel sparse matrices,
\vsp
\item \code{PC} - a collection of preconditioners,
\vsp
\item \code{KSP} - data-structure-neutral implementations of
      many popular Krylov subspace iterative methods,
\vsp
\item \code{SLES} - a higher-level interface for the solution of
      large-scale linear systems, and
\vsp
\item \code{SNES} - data-structure-neutral implementations of Newton-like
      methods for nonlinear systems.
\end{itemize}
\vsp

\section*{Applications}
\code{PETSc} is intended for use in large-scale application projects, and
several ongoing computational science projects at Argonne
and other institutions are built around the \code{PETSc} framework.
With strict attention to component interoperability, \code{PETSc}
facilitates the integration of independently developed application
modules, which often most naturally employ different coding styles and
data structures. 

\section*{Computational Environment}
\code{PETSc} is available on Sun, DEC Alpha, Silicon Graphics, HP-UX, and IBM
RS/6000 workstations; IBM PCs running FreeBSD or Linux; the Cray T3D,
the Intel Paragon; and the IBM SP.

\section*{Availability}

The \code{PETSc} package is freely available.
The complete distribution can be obtained by anonymous ftp from 
\URL{ftp://info.mcs.anl.gov/pub/petsc}.
The \code{PETSc} distribution contains all source code, installation
instructions, a users guide, {\tt man pages} for all routines, and a
collection of examples.  Additional information is available via the
World Wide Web at
\URL{http://www.mcs.anl.gov/petsc/petsc.html}.

\vspace{-.1cm}
\contact{Satish Balay, William Gropp, Lois Curfman McInnes, Barry Smith}
\email{petsc-maint@mcs.anl.gov}
\phone{(708) 252-4318}
\makeinfo
\end{document}
