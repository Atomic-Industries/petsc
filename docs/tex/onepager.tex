% $Id: onepager.tex,v 1.11 1995/10/31 22:23:38 curfman Exp curfman $ 
\documentstyle[epsf,sty/handpage,/home/gropp/hyper/style/anlhtext]{article}
\pagestyle{empty}
\begin{document}
\pagestyle{empty}
\title{PETSc 2.0 for MPI}
\thanks{Mathematics and Computer Science Division,
Argonne National Laboratory,
Argonne, IL 60439-4801.
This work was supported by the Mathematical,
        Information, and Computational Sciences Division subprogram of
        the Office of Computational and Technology Research,
        U.S. Department of Energy, under Contract W-31-109-Eng-38.}

\date{}
\maketitle

\newcommand{\vsp}{\vspace{-1.5mm}}

\section*{Description}

\code{PETSc}, the Portable, Extensible Toolkit for Scientific computation,
is a suite of uni- and parallel-processor codes that are intended for
the solution of large-scale problems modeled by partial differential
equations.  \code{PETSc} employs the MPI standard for all
message-passing communication.  The code is written in a
data-structure-neutral manner to enable easy reuse and flexibility.

\code{ PETSc} is easy to use for beginners.  Moreover, its careful
design allows advanced users to have detailed control over the solution
process. \code{PETSc} integrates a hierarchy of components, all of
which are naturally used in parallel from C, C++, and Fortran, thus
enabling the user to employ the level of abstraction that is most
natural for a particular problem.  Some of the components include:

\vspace{-.4cm}
\begin{itemize}
\item \code{Mat} - a large suite of data structures and code
      for the manipulation of parallel sparse matrices,
\vsp
\item \code{PC} - a small, but growing, collection of preconditioners,
\vsp
\item \code{KSP} - data-structure-neutral implementations of
      many popular Krylov space iterative methods,
\vsp
\item \code{SLES} - a higher-level interface for the solution of
      large-scale linear systems,

\vsp
\item \code{SNES} - data-structure-neutral implementations of Newton-like
      methods for nonlinear systems.
\vsp
\item In addition, we are developing code for manipulating grids
      and discretizations in parallel.
\end{itemize}
\vspace{-.3cm}

\section*{Applications}
\code{PETSc} is intended for use in large-scale application projects, and
several ongoing multi-site computational science projects at Argonne and
other institutions are built around the \code{PETSc} framework.
With strict attention to component interoperability, \code{PETSc}
facilitates the integration of independently developed modules, which
often most naturally employ different coding styles and data
structures.  In addition, users can seemlessly replace models as well
as combine and nest algorithms.

\section*{Computational Environment}
\code{PETSc} is available on Sun, DEC alpha, Silicon Graphics, HP-UX, and IBM
RS/6000 workstations; IBM PCs running freeBSD or Linux; the Cray T3D,
the Intel Paragon; and the IBM SP. Other machines may be supported as our
resources permit.

\section*{Availability}

The \code{PETSc} package is freely available.
The complete distribution can be obtained by anonymous ftp from 
\URL{ftp://info.mcs.anl.gov/pub/petsc}.

\section*{Documentation}

The \code{PETSc} distribution contains all source code, 
installation instructions,
a users guide in both PostScript and HTML formats, 
{\tt man pages} for all routines,
and a collection of examples.
Additional information is available via the World Wide Web at
\URL{http://www.mcs.anl.gov/petsc/petsc.html}.
\contact{William Gropp, Lois Curfman McInnes, Barry Smith}
\email{petsc-maint@mcs.anl.gov}
\phone{(708) 252-4318}
\makeinfo
\end{document}
