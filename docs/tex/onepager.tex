\documentstyle[epsf,sty/handpage,/home/gropp/hyper/style/anlhtext]{article}
\pagestyle{empty}
\begin{document}
\pagestyle{empty}
\title{PETSc 2.0 for MPI}
\thanks{Mathematics and Computer Science Division,
Argonne National Laboratory,
Argonne, IL 60439-4801.
This work was supported by the Mathematical,
        Information, and Computational Sciences Division subprogram of
        the Office of Computational and Technology Research,
        U.S. Department of Energy, under Contract W-31-109-Eng-38.}

\date{}
\maketitle

\section*{Description}

\code{PETSc}, the Portable, Extensible Toolkit for Scientific computation,
is a library for portable, parallel (and serial)
scientific computation that employs the MPI standard for
message-passing communication.  Most of the code is written in a
data-structure-neutral manner to enable easy reuse and flexibility.

\code{PETSc} includes several components, all of which
are naturally and easily used in parallel from C, C++, and Fortran:

\vspace{-.4cm}

\begin{itemize}
\item \code{Mat} - A large suite of data structures and code
      for the manipulation of parallel sparse matrices.
\item \code{PC} - A small, but growing collection of preconditioners.
\item \code{KSP} - Data-structure-neutral implementations of
      many popular Krylov space iterative methods.
\item \code{SLES} - A higher-level interface to a variety
      of preconditioners and Krylov space methods.
\item \code{SNES} - Data-structure-neutral 
      implementations of trust region and line search Newton's 
      methods for nonlinear systems. 
\item In addition, we are developing code for manipulating grids
      and discretizations in parallel.
\end{itemize}

\section*{Applications}
\code{PETSc} is intended for use in large-scale application projects. 
Several ongoing computational science projects at Argonne and
other institutions are built around the \code{PETSc} framework.

\section*{Computational Environment}
\code{PETSc} is available on Sun, DEC, Silicon Graphics, HP-UX, and IBM
RS/6000 workstations; IBM PCs running freeBSD; and the
IBM SP. Other machines will be supported as our resources permit.

\section*{Availability}

The \code{PETSc} package is freely available.
The complete distribution can be obtained by anonymous ftp from 
\URL{ftp://info.mcs.anl.gov/pub/petsc}.

\section*{Documentation}

The \code{PETSc} distribution contains all source code, 
installation instructions,
a users guide in both PostScript and HTML formats, 
{\tt man pages} for all routines,
and a collection of examples.
Additional information is available via the World Wide Web at
\URL{http://www.mcs.anl.gov/petsc/petsc.html}.
\contact{William Gropp, Lois Curfman McInnes, Barry Smith}
\email{petsc-maint@mcs.anl.gov}
\phone{(708) 252-4318}
\makeinfo
\end{document}
