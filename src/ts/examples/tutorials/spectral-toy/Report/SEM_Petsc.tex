\documentclass[10pt]{article}
%\documentclass[review]{siamart0516}
%\usepackage[utf8]{inputenc}
\usepackage[T1]{fontenc}
\usepackage{lmodern}
\usepackage{subfig}
%\usepackage[latin1]{inputenc}
\usepackage[english]{babel}
\usepackage{fullpage}
\usepackage{multirow}
%\usepackage{amsmath,amssymb,psfrag, amsthm}
\usepackage{amsmath,amssymb,psfrag}
\usepackage{graphicx}
\usepackage{listings}
\usepackage{paralist}
\usepackage{appendix}
%\usepackage{natbib}
\usepackage{amsfonts}
\usepackage{subfig,color}
\usepackage{comment} 
\usepackage{enumerate}
\usepackage{cancel}
%\usepackage{helvet}
%\usepackage{algorithm}
%\usepackage{algorithmicx}
%\usepackage{algpseudocode}
\usepackage{epstopdf}
\usepackage{tikz,tikz-cd}
\usepackage{pgfplots,pgfplotstable}
\usetikzlibrary{pgfplots.groupplots}
\usetikzlibrary{positioning}
\usetikzlibrary[shapes,arrows,trees]
\usetikzlibrary{matrix,decorations.pathmorphing}
%\newcommand{\refalg}[1]{Algorithm~\ref{#1}}
%\newcommand{\refsec}[1]{Section~\ref{#1}}
%\newcommand{\reffig}[1]{Figure~\ref{#1}}
%\newcommand{\refsubfig}[1]{Figure~\subref{#1}}
%\newcommand{\reftab}[1]{Table~\ref{#1}}
%\newcommand{\refeqn}[1]{(\ref{#1})}
%\newcommand{\reflst}[1]{Listing~(\ref{#1})}
\newcommand{\refalg}[1]{\cref{#1}}
\newcommand{\refsec}[1]{\cref{#1}}
\newcommand{\reffig}[1]{\cref{#1}}
\newcommand{\refsubfig}[1]{\cref{#1}}
\newcommand{\reftab}[1]{\cref{#1}}
\newcommand{\refeqn}[1]{\cref{#1}}
\newcommand{\reflst}[1]{\cref{#1}}
\renewcommand{\d}{\mathrm{d}}
\newcommand{\re}[1]{(\ref{#1})}
\newcommand{\D}{\mathrm{D}}
\newcommand{\bb}[1]{\boldsymbol{#1}}
\def\sgn{\mathop{\rm sgn}}
\newcommand{\remark}[1]{{\color{red} #1}}
% THEOREMS ETC
\newcommand{\vect}[1]{\mathbf{#1} }
\newcommand{\order}[1]{\mathcal{O}(h^{#1})}
\newcommand{\code}[1]{{\tt #1}}
\renewcommand{\familydefault}{\sfdefault}
% -----------------------------------------------------------
% -----------------------------------------------------------
\lstset{backgroundcolor=\color[rgb]{0.92,0.95,1}}
\lstset{rulecolor=\color[rgb]{0.92,0.95,1}}
\lstset{numbers=left}
\lstset{basicstyle=\ttfamily\footnotesize}
\lstset{numberstyle=\footnotesize}

\def\dfdd#1#2{\frac{\partial#1}{\partial#2}}
\newcommand{\erf}{\, \mathrm{erf}}
\newcommand{\erfc}{\, \mathrm{erfc}}
%\renewcommand{\labelenumi}{[\arabic{enumi}]}


%\normalsize

\begin{document}
%\def\thefigure{\arabic{figure}}
%\def\thetable{\arabic{table}}

% TITLE AND AUTHOR --------------------------------
\title{PDE constrained optimization in spectral element framework using PETSc and TAO}
\vspace{1cm}
\author{Oana Marin, Emil Constantinescu, Barry Smith}
\date{\today}
\maketitle


\section{Prerequisites}
Consider a time dependent partial differential equation with constant coefficients, given by the spatial operator $ \mathbb L(\mathbf u)$ and initial conditions $\mathbf g(\cdot)$ as
\begin{eqnarray}
 \frac{\partial\mathbf u}{\partial t} &=&  \mathbb L(\mathbf u) \ , \mathbf x \in \Omega \\ \nonumber
  \mathbf u|_{\partial \Omega}&=&0, \\ 
  \mathbf u(0,t)&=&\mathbf g(\mathbf x) \ . \nonumber \label{eq:generic}
\end{eqnarray} 

To transtion from the strong form of the equation to the weak form we define the spaces
$$
L^2(\Omega)=\lbrace f :\Omega\rightarrow \mathbb R |\quad \bigg(\int_{\Omega} |f|^2 \d \Omega \bigg)^{1/2}<\infty,\ \rbrace
$$
with the subset
$$
H^1(\Omega)=\lbrace f \in L^2(\Omega) |\quad \frac{\partial f}{\partial \vect x} \ \text{in}\  L^2(\Omega),\ \rbrace
$$
and the space
$$
H^1_0(\Omega)=\lbrace f \in H^1(\Omega) |\quad \frac{\partial f}{\partial \vect x}|_{\partial\Omega}=0 ,\ \rbrace
$$

For the weak form of a partial differential equation we seek all functions with the property that the residual is orthogonal to the set of all test functions.
Consider the solution $\mathbf u$ and test function $\mathbf v$ such that
$(\mathbb L(\vect u),\mathbf v)=0$ for all $\mathbf v$ in a given test space, under the inner product defined as
$$(\mathbf u,\mathbf v)=\int_{\Omega} \mathbf u\ \mathbf v \ \d \Omega$$

To take a concrete example let us set $\mathbb L(\mathbf u) = \nu \Delta \mathbf u$, where $\nu$ is a parameter which represents the viscosity. To derive the weak form we multiply both sides of the partial differential equation (Eq.~\ref{eq:generic}) by a test function $\mathbf v$ leading to
$$(\frac{\partial\mathbf u}{\partial t},\mathbf v) + (\nu\Delta \mathbf u, \mathbf v) =0$$
In this case  becomes after applying integration by parts
\begin{equation}
\int_{\Omega}\frac{\partial\mathbf u}{\partial t} \mathbf v \ \d \Omega + \int_{\Omega} \nu\nabla \mathbf u\ \nabla \mathbf v\ \d \Omega -
 \int_{\Gamma} \nabla \mathbf u \ \mathbf v\  \d \Gamma =0\label{eq:weakheat}
\end{equation}

The boundary term vanishes since $\mathbf u, \mathbf v \in H^1_0(\Omega)$ and $H^1_0(\Omega)$ is chosen such that it satisfies the boundary conditions.
\section{The Spectral Element Method}
Consider now for simplicty the one dimensional heat equation
\begin{equation}
\frac{\partial u}{\partial t} - \nu\Delta u\ =0\ . \label{eq:1dheat}
\end{equation}


Based on the weak form given by Eq.~\ref{eq:weakheat} several discretizations are suitable, Finite Element Method, Spectral Element, Continous or Discontinous Galerkin, etc.
The Spectral Element Method is a subclass of Galerkin methods, or weighted residual methods. 
The idea is to minimize the error of the numerical computation in the energy norm over a 
chosen space of polynomials. 

In the case of a spectral element method the domain $\Omega=\cup_{e=1,M} \Omega_e $, is decomposed in $M$ nonoverlapping 
subdomains $\Omega_e$, termed elements, over which the data will be represented by orthogonal polynomials.

Let us now define $X=H^1_0(\Omega)^d$. The space of polynomials of order $N$ defined over an element $\Omega^e, \ e=1,\ldots, E$ is
$$
\mathbb P_{N,E}=\lbrace  \phi| \phi \in L^2(\Omega); \quad \phi|_{\Omega^e} \text{polynomial of degree} \leq N\rbrace
$$
Subsequently $X_N=X\cap  P_{N,E}^d$, where $d$ is the dimension of the problem.

In the polynomial space $X_N$ we expand the numerical solution $u(x)=\sum_i^N u_i(x)\phi_i$.
With this choice the solution as well as the test space function $v$ can be expanded as
\begin{equation}
u(\vect x)=\sum_{i=1}^N u_{i}\phi_i(\vect x)
\label{eq:ansatz}
\end{equation}

The point discretization can be either a Chebyshev grid, or a Legendre grid, in either case the polynomials are orthogonal $(\phi_i \phi_j)=\delta_{ij}$. 
For the moment we consider solely one dimensional problems
To proceed with the numerical discretization we plug Eq.~\ref{eq:ansatz} into Eq.~\ref{eq:weakheat} to obtain
\begin{equation}
\frac{\partial}{\partial t}\int \sum_{i=1}^N u_{i}\phi_i(x) \sum_{j=1}^N v_{j}\phi_j(x) \d \Omega_e + \int\sum_{i=1}^N u_{i}\phi'_i(x) \sum_{j=1}^N v_{j}\phi'_j(x) \d \Omega_e  =0 \ .
\end{equation}

With proper reordering this yields

\begin{equation}
\frac{\partial}{\partial t}\sum_{i=1}^N u_{i}\sum_{j=1}^N  v_{j}\underbrace{\int \phi_i(x)\phi_j(x) \d \Omega_e}_{M_{ij}} + \sum_{i=1}^N u_{i}\sum_{j=1}^N v_{j}\underbrace{\int\phi'_i(x) \phi'_j(x) \d \Omega_e}_{K_{ij}}  =0
\end{equation}

where $M_{ij}$ is the mass matrix and $K_{ij}$ is the stiffness matrix.

In algebraic form we can rewrite
\begin{equation}
\frac{\partial}{\partial t}\underline{v}^TM\underline{u} + \underline{v}^TK\underline{u}=0
\end{equation}

with $\underline{u}=(u_0,\ u_1, \ldots, u_N)$ and similarly $\underline v$ is the array of point values of the test function. 
Note that one can factor out $v$ to obtain an ODE

\begin{equation}
\frac{\partial}{\partial t} M \underline{u}=- K \underline{u} \label{eq:discrete}
\end{equation}

If the domain $\Omega_e \neq [-1, \ 1]$ a change of variables is in order.
Presume  $\Omega_e = [a, \ b]$ and define the reference element $\hat{\Omega} = [-1, \ 1]$.
For $x\in \Omega_e$ and $r \in \hat{\Omega}$ the mapping is
$$x= a+ \frac{b-a}{r+1}$$

%And the change of variables reads
%$$\frac{dx}{dr
\begin{comment}
\section{ODE solutions}
\subsection{Convergence tests}
\subsection{Various Boundary conditions}

\subsection{Convection-Diffusion}
Consider the energy equation in non-dimensional form Eq.~\ref{eq:energy_nondim}, this is a convection diffusion equation with the Laplacian term weighted by the P\' eclet number. For the simplicity of the presentation we make two changes 
\begin{itemize}
\item replace the advection velocity $\vect u$ by a constant velocity field $c$ that is divergence free
\item denote the temperature $T$ in the energy equation by $u$
\end{itemize}
The variational formulation of the Convection-Diffusion equation reads
\begin{eqnarray}\label{eq:Conv_diff}
\int_{\Omega}\frac{\partial u}{\partial t} \cdot v\d \Omega+ \int_{\Omega}(c \cdot \nabla u)\cdot  v\d \Omega\ &=& \frac{1}{Pe}\int_{\Omega} (\nabla\cdot \nabla  u)\cdot  v\d \Omega + \int_{\Omega}f\cdot v\d \Omega.
\end{eqnarray}
with $ u,\ v \ \in \  H^1_0(\Omega)$.

The terms in the convection diffusion equation can be further expanded to give
\begin{eqnarray}\label{eq:var_cd}
\int_{\Omega}\frac{\partial u}{\partial t} \cdot \vect v\d \Omega &=&\frac{\partial}{\partial t} \int_{\Omega}u \cdot \vect v\d \Omega\\
\int_{\Omega} \nabla\cdot \nabla u\cdot v\d \Omega &=&\int_{\Omega}  \nabla u\cdot \nabla v\d \Omega +\int_{\partial\Omega} v \nabla u \cdot \vect n\d\Omega
\end{eqnarray}


Let us now define $X=H^1_0(\Omega)^d$. The space of polynomials of order $N$ defined over an element $\Omega^e, \ e=1,\ldots, E$ is
$$
\mathbb P_{N,E}=\lbrace  \phi| \phi \in L^2(\Omega); \quad \phi|_{\Omega^e} \text{polynomial of degree} \leq N\rbrace
$$
Subsequently $X_N=X\cap  P_{N,E}^d$, where $d$ is the dimension of the problem.

In the polynomial space $X_N$ we expand the numerical solution $u=\sum_i^N u_i\phi_i$.
With this choice the solution as well as the test space function $\vect v$ can be expanded as
$$
u(\vect x)=\sum_{i=1}^N u_{i}\phi_i(\vect x)
$$

Given the choice of the polynomial space $X_N$ the boundary integral term in \ref{eq:var_cd} vanishes. Upon insertion of the ansatz into the variational formulation we have
\begin{eqnarray}\label{eq:cd_ansatz}
\int_{\Omega}\frac{\partial u}{\partial t} \cdot \vect v\d \Omega &=&\frac{\partial}{\partial t} \sum_i^N\sum_j^N v_i\bigg(\int_{\Omega}  \phi_i(\vect x) \cdot \phi_j(\vect x)\d \Omega u_j\bigg) \\
\int_{\Omega}(c \cdot \nabla u)\cdot  v\d \Omega\ &=&\sum_i^N\sum_j^N v_i\bigg (\int_{\Omega}(c \cdot \phi_i(\vect x)\cdot \nabla\phi_j(\vect x)  \d \Omega\bigg) u_j\ \\
\int_{\Omega} (\nabla\cdot \nabla  u)\cdot  v\d \Omega &=&\sum_{i=1}^N \sum_{j=1}^N v_{i}(\int_{\Omega}\nabla\phi_i(\vect x) \cdot \nabla \phi_j(\vect x)\d \Omega)u_j \\
\end{eqnarray}

To discretize the integrals in \ref{eq:cd_ansatz} we need to introduce a quadrature rule. For spectral accuracy the choice is the Gauss-Legendre quadrature
$\int_{-1}^1\phi(\vect x)\d x=\sum_k \rho_k\phi(x_k)$
where the quadrature points $x_k$ are given by the Gauss-Legendre-Lobatto points, and the weights $\rho_k$ are based on the Legendre polynomials as in \ref{dfm02}.
To start with we consider the one dimensional case and proceed to higher dimensions and curvilinear elements.

\subsubsection{One Dimensional case}
Let us first regard a one dimensional case on the domain $\Omega=[a\ b]$ and analyse one by one the terms in Eq.~\ref{eq:cd_ansatz}. Since the quadrature rule is defined on the interval $[-1, \ 1]$ we map $x \in \Omega$ to $r\in [-1,\ 1]$ via $x=a+(b-a)(r+1)/2$ and take $L=b-a$.
The mass matrix is given by 
\begin{equation}
M_{ij}=\int_{\Omega}\phi_i(x) \phi_j(x)\d \Omega=\frac{L}{2}\int_{-1}^1\phi_i(r) \phi_j(r)\d r=\frac{L}{2}\sum_k \rho_k\phi_i(r_k) \phi_j(r_k)
\end{equation}
since $\phi_i(x_k) \phi_j(x_k)=\delta_{ij}$, $M$ is a diagonal matrix which holds on the diagonal the weights of the quadrature rule. In practice $M$ can also be used as an integration operator, therefore one can write $\int_{-1}^1f(x)\d x\approx Mf$

The stiffness matrix corresponding to the second order term is 
\begin{equation}
A_{ij}=\int_{\Omega}\phi'_i(\vect x) \phi'_j(\vect x)\d \Omega=\frac{L}{2}\int_{-1}^1\phi'_i(r) \phi'_j(r)\d r=\frac{L}{2}\sum_k \rho_k\phi'_i(r_k) \phi'_j(r_k)
\end{equation}
and the convection operator
\begin{equation}
C_{ij}=\int_{\Omega}c\phi_i(\vect x) \phi'_j(\vect x)\d \Omega=\frac{L}{2}\int_{-1}^1c(r)\phi_i(r) \phi'_j(r)\d r=\frac{L}{2}\sum_k c(r_k)\rho_k\phi_i(r_k) \phi'_j(r_k)
\end{equation}

This leads to the following spatial discretized system of equations
\begin{equation}\label{eq:advd}
M \frac{d u}{dt} = \, Au -Cu + M f, 
\end{equation}


\subsubsection{Two Dimensional case}
Consider $\Omega=[-1,\ 1]^2$ discretized in $N$ GLL points. Then the basis function $\pi_k(x_1,x_2)=\phi_i(x_1)\phi_j(x_2)$ where $i,j=1,\ldots,N$ and $k=i+(N+1)\cdot j$.

The ansatz on the solution 
$$u(x,y)=\sum_{i=0}^M\sum_{j=0}^N u_{ij}\phi_i(x)\phi_j(y)$$

To start with we introduce a notation for tensor product form of matrix multiplication
\begin{equation}
w_{ij}=\sum_{l=1}^M\sum_{k=1}^N a_{jl}b_{ik}u_{kl}
\end{equation}
In the tensor product notation $c_{ij}=\sum_{l=1}^M\sum_{k=1}^N a_{jl}b_{ik}$ can be written as $C=A\otimes B)$. By unrolling the vector $w_{ij}$ as $\underline{w}_{\hat{i}}$ with components given by the ordering $\hat{i}=i+M\cdot(j-1)$ and similarly for $u_{kl}$ we can expand the tensor matrix product as \footnote{this doesn't match the one before.. careful with N+1, N}

$$\underline{w}=(A\otimes I)(I\otimes B)\underline{u}$$

From this notation simpler cases arise such as
\begin{equation}
w_{ij}=\sum_{k=1}^N a_{ik}u_{kj}\quad \rightarrow \quad \underline{w}=
\end{equation}
\section{Data Structure for SEM in Petsc}
\label{sec:sem}


%\begin{equation}
%  \mathbf u(\mathbf x)|_{\Omega_e} = \sum_{i,j,k} u^e_{ijk}
%  \phi_i(x_1)\phi_j(x_2)\phi_k(x_3)\ ,
%\end{equation} 

\label{sec:Spectral and Pseudo-spectral}
\end{comment}

\section{Inverse problem}

Consider again the forward partial differential equation describing heat transfer
\begin{eqnarray}
 \frac{\partial u}{\partial t} - \nu\Delta  u  &= &0, \  x \in \Omega \\ \nonumber
   u|_{\partial \Omega}&=& u_b, \\ \nonumber
   u( x,0)&=& u_0( x)\label{eq:heateq}
\end{eqnarray} 

As a generic notation we use for the PDE
$$\mathcal P [ u]=0$$. 

Presume we seek to find the intial condition $u_0$ that leads at time $T$ a solution $\mathbf u$ which matches a reference solution $\mathbf u_d$. Typically $ u_d$  is not known over the entire domain $\Omega$ in practice, but in a subset region $\Omega_s$ where sensors have been placed. However for simplicity here we construct a continous function $\mathbf u_d$ to avoid regularization issues.

The approach is to construct a cost function which minimizes the difference between $| u- u_d|$ while preserving convexity.
$$J[u|_T]=\int_{\Omega}(u(T)- u_d(T))^2 \ \d \Omega  $$

We seek $ u$ solution to Eq.~\ref{eq:heateq} such that $J[ u]$ is minimal.
\subsection{Continous Adjoint - Variational calculus derivation}

To derive the gradient used in the backward Set up the Lagrangian
$$\mathcal{L}[ u,  v]=\int_{\Omega}( u[T]- u_d[T])^2 \ \d \Omega +\int_0^T\int_{\Omega} P[ u]  v \ \d \Omega \d t + \int_0^T\int_{\Gamma}  v_b  ( u-  u_b) \ \d \Gamma \d t$$

\footnote{Chain rule for $f=g\circ h$, g must be Frechet and if h' is Frechet or Gateaux so is f.}

\footnote{Riesz theorem: Let T be a functional on H then the Frechet derivative $T'(x)$ is given by
$T'(x) v= (\nabla T, v)$ for $\forall v \in H$.}

Rewrite of the term
\begin{eqnarray}
\int_0^T\int_{\Omega} P[ u]  v \ \d \Omega \d t &=&
\int_0^T\int_{\Omega} \frac{\partial u}{\partial t}  v \ \d \Omega \d t+ \int_0^T\int_{\Omega} \nu\Delta  u \cdot  v \ \d \Omega \d t+\int_0^T\int_{\Omega}  f( x)   v \ \d \Omega \d t \\ \nonumber
&=& \int_{\Omega} u v \ \d \Omega|_0^T -
\int_0^T\int_{\Omega}
 \frac{\partial v}{\partial t}  u \ \d \Omega \d t+ \int_0^T\int_{\Gamma}(\nabla  u\cdot  v-  u\cdot \nabla v) n\ \d \Gamma \d t + \\
 && \int_0^T\int_{\Omega} \nu\Delta  v\cdot   u \ \d \Omega \d t+
 \int_0^T\int_{\Omega}  f( x)   v \ \d \Omega \d t \\ \nonumber
 &=& 
\int_0^T\int_{\Omega}
(- \frac{\partial v}{\partial t}  +\nu\Delta  v)\cdot   u\ \d \Omega \d t+\int_{\Omega} u v |_0^T \ \d \Omega
+ \int_0^T\int_{\Gamma}(\nabla  u\cdot  v-  u\cdot \nabla v) n\ \d \Gamma \d t \\
&&+
 \int_0^T\int_{\Omega}  f( x)   v \ \d \Omega \d t \\ \nonumber
\end{eqnarray}

Grouping now the time dependent terms we have
\begin{eqnarray}
\mathcal{L}[ u, u_0,  v]&=&\int_{\Omega}( u[T]- u_d[T])^2 + u[T] v[T]\ \d \Omega -\int_{\Omega} u[0] v[0]\ \d \Omega+\int_0^T\int_{\Omega} \overline{P}[ v]  u \ \d \Omega \d t + \\ \nonumber
 && \int_0^T\int_{\Gamma}  v_b  ( u-  u_b) \ \d \Gamma \d t +\int_0^T\int_{\Gamma}(\nabla  u\cdot  v-  u\cdot \nabla v) n\ \d \Gamma \d t +
 \int_0^T\int_{\Omega}  f( x)   v \ \d \Omega \d t 
\end{eqnarray}

\begin{eqnarray}
\frac{\partial L}{\partial  u}&=&- \frac{\partial v}{\partial t}  +\nu\Delta  v \\
\frac{\partial L}{\partial  u_0}&=&- v[0]\\
\frac{\partial L}{\partial  u_T}&=&2( u[T]- u_d[T])+ v[T]\\
\frac{\partial L}{\partial  u_b}&=&\\
\end{eqnarray}

\subsection{Discrete Adjoint}

\section{Validation and sources of error}

Sources of error
\begin{itemize}
\item Accumulation of local truncation error in the timestepper
\item Inaccurate gradient information may hinder convergence


\end{itemize}



%
%\begin{figure}
%\begin{center}
% \begin{tikzpicture}[scale=2.0]
%\node (A) at (0,1) {$\mathbf u$};
%\node (B) at (1.5,1) {$T\mathbf u$};
%\node (C) at (1.5,0) {$T\tilde{\mathbf u}$};
%\node (D) at (0,0) {$\tilde{\mathbf u}$};
%\path[->,font=\scriptsize,>=angle 90]
%(A) edge node[above]{$DCT$} (B)
%(D) edge node[right]{$Error$} (A)
%(B) edge node[right]{$Truncate \rightarrow \texttt{Huffman\ encode}$} (C)
%(C) edge node[above]{$IDCT$} (D);
%\end{tikzpicture}
%\end{center}
% \caption{Workflow for compression and restoring the solution.} 
%  \label{fig:algorithm}
%\end{figure}
%
%\subsection{Algorithm}
%\begin{algorithm}
%\begin{algorithmic}[5]
%\Procedure{Setup}{}
%\State  {\tt T $\leftarrow$ DCT\_setup\_1d(p) }
%%\State {\tt M $\leftarrow$ build\_map($\text{mesh}_{\text{gll}})$}
%\EndProcedure
%\Procedure{Truncate}{}
%%\For {\tt k$\in$ $LocalEL_{id}(COMPnode)$} %\Comment{$ttotal$}
%\For { {\tt 0 $\leq$ k $<$ nel} }%\Comment{$ttotal$}
%%\State $u=M\cdot u$
%\State  {\tt $\text{u}_{\text{DCT,k}}=\text{T}\cdot \text{u}_\text{k} \cdot
%\text{T}^T \cdot \text{T}^T$} 
%\State { sort($\text{u}_{\text{DCT,k}}$)}
%%\State $\mathrm{u_{trunc}}=\mathrm{u_{DCT}}>\epsilon$
%\State {\tt$\text{u}_{\text{trunc,k}}=\text{u}_{\text{DCT,k}}>\epsilon$}
%\EndFor
%\EndProcedure
%\Procedure{Compress}{}
%\If {\tt IOnode} 
%\For {\tt q $\in$ IOchildren}
%\State {\tt Recv($\text{u}_{\text{trunc}}$,q)}
%\State {\tt $\text{u}_{\text{compress}}=$huff\_encode($\text{u}_{\text{trunc}}$)}
%\State {\tt output($\text{u}_{\text{compress}}$)}
%\EndFor
%\Else
%\State {\tt Send($\text{u}_{\text{trunc}}$,IOparent)}
%\EndIf
%\EndProcedure
%\end{algorithmic}
%\caption{Parallel compression on an already partitioned mesh with {\tt nel} elements
%each.}
%\label{alg:code_struct}
%\end{algorithm}
%
%\begin{figure}[!ht]
%\centering
%\subfloat[Compression ratio ($C_r$) vs error: (blue) GLL grid, (black) Chebyshev grid, 
%(green marker) corresponds to visualization in \reffig{fig:wingvis97}.]
%{\includegraphics[width=0.47\textwidth]{data/compvserrwing.eps}
%\label{fig:wingfull_err}}
%\quad
%\subfloat[A priori vs a posteriori error: (blue) GLL grid, (black) Chebyshev
%grid.]
%{\includegraphics[width=0.47\textwidth]{data/overestwing.eps}
%\label{fig:wingfull_est}}
%\caption{Flow past an airplane wing}
%\end{figure}

\section{Conclusions}
\label{sec:conclusion}


\section*{Acknowledgments}

This research used resources of the Argonne Leadership Computing Facility, 
which is a DOE Office of Science User Facility supported under Contract 
DE-AC02-06CH11357.


\vskip 10pt
%\begin{flushright}
\noindent\scriptsize \framebox{
%\parbox{3.2in}{
\parbox{0.96\textwidth}{
The submitted manuscript has been created by the University of Chicago
as Operator of Argonne National Laboratory (``Argonne'') under
Contract No. DE-AC02-06CH11357 with the U.S. Department of Energy.
The U.S. Government retains for itself, and others acting on its
behalf, a paid-up, nonexclusive, irrevocable worldwide license in said
article to reproduce, prepare derivative works, distribute copies to
the public, and perform publicly and display publicly, by or on behalf
of the Government.
}
} \normalsize
\newpage

\bibliographystyle{siamplain}
\bibliography{SEM_Petsc}
\end{document}

\begin{comment}

Set up the Lagrangian
$$\mathcal{L}[\vect u, \vect u_0]=\int_{\Omega}(\mathbf u[T]-\mathbf u_d[T])^2 \ \d \Omega +\int_0^T\int_{\Omega} P[\mathbf u] \mathbf v \ \d \Omega \d t + \int_0^T\int_{\Gamma} \mathbf v_b \mathbf (\mathbf u- \mathbf u_b) \ \d \Gamma \d t$$

Note: Chain rule for $f=g\circ h$, g must be Frechet and if h' is Frechet or gateaux so is f.

Note2: Riesz theorem: Let T be a functional on H then the Frechet derivative $T'(x)$ is given by
$T'(x) v= (\nabla T, v)$ for $\forall v \in H$.

Rewrite of the term
\begin{eqnarray}
\int_0^T\int_{\Omega} P[\mathbf u] \mathbf v \ \d \Omega \d t &=&
\int_0^T\int_{\Omega} \frac{\partial\mathbf u}{\partial t} \mathbf v \ \d \Omega \d t+ \int_0^T\int_{\Omega} \nu\Delta \mathbf u \cdot \mathbf v \ \d \Omega \d t+\int_0^T\int_{\Omega} \mathbf f(\mathbf x)  \mathbf v \ \d \Omega \d t \\ \nonumber
&=& \int_{\Omega}\mathbf u\mathbf v \ \d \Omega|_0^T -
\int_0^T\int_{\Omega}
 \frac{\partial\mathbf v}{\partial t} \mathbf u \ \d \Omega \d t+ \int_0^T\int_{\Gamma}(\nabla \mathbf u\cdot \mathbf v- \mathbf u\cdot \nabla\mathbf v)\mathbf n\ \d \Gamma \d t +
 \int_0^T\int_{\Omega} \nu\Delta \mathbf v\cdot  \mathbf u \ \d \Omega \d t+
 \int_0^T\int_{\Omega} \mathbf f(\mathbf x)  \mathbf v \ \d \Omega \d t \\ \nonumber
 &=& 
\int_0^T\int_{\Omega}
(- \frac{\partial\mathbf v}{\partial t}  +\nu\Delta \mathbf v)\cdot  \mathbf u\ \d \Omega \d t+\int_{\Omega}\mathbf u\mathbf v |_0^T \ \d \Omega
+ \int_0^T\int_{\Gamma}(\nabla \mathbf u\cdot \mathbf v- \mathbf u\cdot \nabla\mathbf v)\mathbf n\ \d \Gamma \d t +
 \int_0^T\int_{\Omega} \mathbf f(\mathbf x)  \mathbf v \ \d \Omega \d t \\ \nonumber
\end{eqnarray}

Grouping now the time dependent terms we have
\begin{eqnarray}
\mathcal{L}[\vect u, \vect u_0, \mathbf v]&=&\int_{\Omega}(\mathbf u[T]-\mathbf u_d[T])^2 +\mathbf u[T]\mathbf v[T]\ \d \Omega -\int_{\Omega}\mathbf u[0]\mathbf v[0]\ \d \Omega+\int_0^T\int_{\Omega} \overline{P}[\mathbf v] \mathbf u \ \d \Omega \d t + \\ \nonumber
 && \int_0^T\int_{\Gamma} \mathbf v_b \mathbf (\mathbf u- \mathbf u_b) \ \d \Gamma \d t +\int_0^T\int_{\Gamma}(\nabla \mathbf u\cdot \mathbf v- \mathbf u\cdot \nabla\mathbf v)\mathbf n\ \d \Gamma \d t +
 \int_0^T\int_{\Omega} \mathbf f(\mathbf x)  \mathbf v \ \d \Omega \d t 
\end{eqnarray}

\begin{eqnarray}
\frac{\partial L}{\partial \mathbf u}&=&- \frac{\partial\mathbf v}{\partial t}  +\nu\Delta \mathbf v \\
\frac{\partial L}{\partial \mathbf u_0}&=&-\mathbf v[0]\\
\frac{\partial L}{\partial \mathbf u_T}&=&2(\mathbf u[T]-\mathbf u_d[T])+\mathbf v[T]\\
\frac{\partial L}{\partial \mathbf u_b}&=&\\
\end{eqnarray}


\end{comment}